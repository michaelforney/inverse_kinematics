\documentclass{article}

\usepackage[margin=1in]{geometry}
\usepackage{setspace}
\usepackage{graphicx}
\usepackage{amsmath}

\title{Final Project Report}
\author{Michael Forney \\ SID: 21392560}

\begin{document}
    \maketitle
    \doublespacing

    \section{Project Overview}
    My final project is a interactive inverse kinematics solver, which can
    generate simple animations by interpolating poses created manually by the
    user, or automatically through inverse kinematics. I used a variation of the
    cyclic coordinate descent algorithm to iteratively improve the parameters of
    the joints, gradually converging on the best solution given a certain set of
    constraints. My project can solve joint structures with arbitrary branching
    at each socket (only outward branching, joints cannot reconnect further down
    in the tree), and both revolute (rotational), and prismatic (translational)
    joints.

    \section{Cyclic Coordinate Descent}

    \section{Calculating Joint Parameters}

    \subsection{Revolute Joints}
    In order to minimize the total distance between target end points $t_i$
    and current end points (rotated by some angle $\theta$), $e_i$, I first
    expanded the dot product to simplify the problem.

    \begin{align}
        error &= \sum_{i=1}^n ||t_i - M(\theta) e_i||^2 \\
              &= \sum_{i=1}^n (t_i - M(\theta) e_i) \cdot (t_i - M(\theta) e_i) \\
              &= \sum_{i=1}^n t_i \cdot t_i + (M(\theta) e_i) \cdot
                  (M(\theta) e_i) - 2 t_i \cdot (M(\theta) e_i)
    \end{align}

    \(M(\theta)\) is an orthogonal matrix, so \(||M(\theta) e_i|| = ||e_i||\).
    This makes \(t_i \cdot (M(\theta) e_i)\) the only part of the equation that
    depends on \(\theta\). In order to minimise the error with respect to
    $\theta$, we can maximise this dot product. To find the maximum, I found
    where the derivative was $0$.

    \begin{align}
        0 &= 
    \end{align}

    \subsection{Prismatic Joints}

\end{document}

